I wrote the Undergraduate Thesis for my bachelors degree during 2020. It was my first formal research/scientific essay. In the beginning of the year, I decided to give it my best and experience the most, because I would probably not have another chance to try it out, specially with an advisor. I was lucky enough to choose a topic I ended up loving. I spent around five months studying very hard and implementing a lot of ideas I read about. I got a rough sketch of everything I wanted to discuss in my essay in June and started working on the essay soon after.

By November I had a decent looking draft which had all the content I wanted to develop and it was around 80 pages long. I spent the next 2 months revising and improving the text and, although I ended up adding some more content, I got to shrink it to 71 pages. I loved working on it and was very happy that my advisor and the grading committee liked it as well. During the work, that lasted 13 months, I do not remember getting stuck or unfocused.

Leading up to this semester I wanted to do more of that. I reached out to a few professors at RIT and I was very hopeful of working with my current advisor, Stanis\l{}aw P. Radziszowki. Due to academic requirements, the initial research plan was revised to reserve half of the semester to developing a visual tool, which was not exactly what I wanted to do, but I did not mind that.

Focusing on the second half of the semester, I believe that a first mistake I made was in my expectations. I remember thinking that I had a very easy time writing and, unlike my thesis, I would not be starting from scratch: I already knew the proper tools, how to draft a plan that works for me, how to organize references, take good notes an so on. I thought the only missing piece was the knowledge, so after starting learning, I would be ready to start writing.

There were two anticipated text deliveries this semester, but the first one got merged into this final report due to my inability to write it. I cannot pinpoint exactly what the problem was but I will list what I believe major factors were. Unlike with Combinatorial Game Theory, that has didactic materials and books that cover all the developments in the field except for some bleeding edge changes, Ramsey Theory is much more opaque.

In one side I had \textit{Winning Ways}, a gem containing every basic idea there is to know presented in a brilliant manner, and on the other \textit{Combinatorial Game Theory} by Aaron Siegel, which formalized every topic and had the notation used essentially everywhere else. Not only that but there was also CGSuite, by Siegel again, that gave visuals and calculations for every prototype or idea I had. This semester was a different process, where I was reaching out for information scattered across papers, that used different notations and terminologies.

Although I believe I learned a lot and was introduced to interesting and fun subjects, the content did not have time to sink in. When writing, this played the role of making me scared of producing false statements or incorrectly quoting authors. I second guessed some decisions I took and ideas I had. It definitely slowed down the process of writing the essay.

For the same reason it was difficult to produce steady progress. There were times when I understood a new idea, maybe in a proof for a theorem, and that just threw a piece I was developing out. This was progress, in the sense that I learned, but was a slow down in the sense that I did not produce anything. This happened once when I was studying the problem of making the painter stronger by increasing the number of edges before coloring and again when I described the work by Šárka Petříčková.

The problems making progress during the middle of the semester led me to the biggest issue I had. Writing the status report became a burden. I found myself constantly thinking about it, but when I sat down to produce I did not have the words or the ideas necessary to make something of reasonable quality. I want to highlight that my advisor never pressured me. He was always understanding and supportive, at times even reassuring about some ideas I wanted to discard. However, I really wanted to have something to show for.

While having disappointments, I know a bit more now about myself and my relation with writing scientific texts. A first lesson is how valuable a general solid understanding of a field is. Usually, because I write about what I know best, I overlooked the fact that, knowing some theorems and notations does not mean you can write a cohesive and interesting text about a subject. There tends to be more to the ideas connecting theorems than the sum of their results.

A second lesson I take is that I am bad at writing on demand. I am usually great with deadlines and become more productive the more urgent something becomes, but it seems I write the best when I am calm. Following up there an advice my advisor gave me early on but which I did not follow. When developing the app he told me to start writing, but I, instead wanted to finish the app as soon as possible. If I had started earlier I would bump into the knowledge holes earlier and get more time to mature my questions.

It does seems starting to write as early as possible is good because, like in software development, you only notice you have questions when you try to do it yourself. I feel weird now about not following the advice, because it does seem very obvious. A fourth lesson is the value a good book has. I always appreciated books, because I am very focused and get a lot from every one I read. Not having them, though, makes it even more evident. There are other lessons, but I think I brought to light the most meaningful to me right now.

To finalize the text I believe I should say I am happy to have taken this class. Although it came with quite some mental burden, which I described abundantly, I loved the experience and I am forever glad I had this experience sooner rather then later. The end results are not what I hoped for, in the writing department, but I am happy about the effort I put into figuring some problems out. Without a doubt, the difficulties became lessons due to the help of my advisor. I think he did a great job incentivising and reassuring me every time I needed that.















