The work done during the semester was split into the development of a visualization tool for simulations involving graphs and the study of Ramsey Theory. The time dedicated to each part was about half of the academic semester, but more details can be found on the activity rundown on Appendix A and hosted in this independent study's main page\footnote{Located on https://people.rit.edu/$\sim$mtl9706/is.}. The final state of the application developed and a copy of this report are also found on the same web page.

The next two sections bring an executive summary of Online Ramsey Theory on Planar Graphs, with what was studied the most, and a description of the simulation tool. The last section is a retrospective of the semester, including expectations and results. The references section mentions papers that are not quoted in the text but were studied and helped better understanding some ideas.

\subsection*{Overview}

\begin{itemize}
	\item \textbf{Course:} CSCI0599 - Computer Science Undergraduate Independent Study.
	\item \textbf{Credit count:} 3 Advanced Elective credits.
	\item \textbf{Term:} Fall of 2021.
	\item \textbf{Advisor:} Stanis\l{}aw P. Radziszowski.
	\item \textbf{Meeting:} 11 meetings throughout the semester.
	\item \textbf{Topic:} Online Ramsey Theory in Planar Graphs.
	\item \textbf{Web page:} https://people.rit.edu/$\sim$mtl9706/is.
	\item \textbf{Github:} https://github.com/MLaurentys/IS\_ramsey.
	\item \textbf{Approved by:} Elouise Oyzon, department of Interactive Games and Media.
\end{itemize}

\subsection*{Objectives}

The main objective of the study was bringing together algebraic results of Ramsey Theory and a graph visualization tool. The choice to focus on planar graphs is tied to the fact that they provide easier visualization with plane embedding. The secondary goal was getting a deeper understanding of the online version of Ramsey type problems, and, more specifically, when restricting the problems to the class of planar graphs.