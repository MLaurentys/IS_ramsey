Up to this point the rules of Ramsey games were Left selecting two disconnected vertices and adding an edge between them and Right coloring it with a color from a predefined set of colors. There is a great difference between results of self-unavoidability and values for Ramsey numbers and Online Ramsey numbers. This difference comes from the lack of knowledge the Right player has, so making Right stronger might close that gap.

In the version presented so far, in each round Left adds an edge and Right colors that edge. Consider, instead, that Left adds two connected edges and Right colors both in their turns. This version is better for Right as it has more information to work with before coloring. However it is not clear if this handicap is enough to close the gap, nor if it makes any significant difference.

Taking k to be the number of edges in the connected component added and colored each turn, denote each version of the Online Ramsey Game to be k-Online Ramsey.

\begin{claim} {1}
:The class $\mathcal{G}$ is unavoidable in $1$-Online Ramsey $\iff$ $\mathcal{G}$ is unavoidable in $k$-Online Ramsey for any positive integer $k$.
\end{claim}
\begin{claim} {2}
:Take $f_k(G, (H_1, \ldots, H_c))$ to be the function that maps a $k$-Online Ramsey game to its respective number. Then, 
$$f_k(G, (H_1, \ldots, H_c)) = g(k) \cdot f_1(G, (H_1, \ldots, H_c))$$
with \mbox{$g \in O(1)$}. Same as saying that the additional information makes no significant difference.
\end{claim}

Evidence:

Consider the theorem by Grytczuk, Haluszczak and Kierstead that the class of forests is self-unavoidable in the $1$-Online Ramsey with $2$ colors. The same property applies to any $k$-Online Ramsey, with the same proof.

Proposition:

The class of forests is self-unavoidable in k-Online Ramsey.

Proof:

It suffices to prove the assertion for trees. Let $T$ be any tree with $n$ vertices $v_i, \ldots, v_n$. We apply induction on the number of vertices. Let $u$ be a leaf of $T$ and let $T' = T - u$. By induction the Builder can force $n$ disjoint copies of $T_1, \ldots, T_n$ of $T'$ in the same color.
Now, in each of the following $n$ rounds, the Builder adds an edge $(u_i, u_j)$, for $u_i \in V(T_i)$ and $u_j \in V(T_j)$ the vertices adjacent to the corresponding vertex $u$ removed. Builder also adds edges $(u_i, v_1), \ldots, (u_i, v_k-1)$, with $v_i$ any vertex disconnected from all trees, in its turn. Regardless of the Painter decision, when the last edge $(u_i, u_j)$ is colored, a monochromatic $T$ is created.

Note that both claims are supported. Claim 2 is supported as the Builder build $k-1$ additional edges in each round, incrementing the total number of needed vertices by a constant factor of $k-1$.


