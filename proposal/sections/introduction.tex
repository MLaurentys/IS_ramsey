The Online Ramsey Game is played alternately between two players, called Builder and Painter, on a graph that is not complete. In their turns, the Builder creates an edge and the Painter paints it. Builder's objective is to create a monochromatic subgraph that is isomorphic to a target graph and Painter's objective is to avoid it as long as possible. The Ramsey Theorem, from which the game is named after, guarantees that if there are enough vertices and the Builder is able to choose edges freely, then he/she always wins, regardless of the target.

Some of the interesting questions come from trying to parameterize the number of moves needed by the Builder and the amount of colors for him/her to win. Other problems arise from limiting rules added to the ones listed above. Consider a class of graphs $\mathcal{G}$, a graph $G$ in $\mathcal{G}$ and a target graph $H$. It is interesting to know whether there exists a sequence of moves starting on $G$ such that all board states $(G_1, \ldots, G_n)$ are inside $\mathcal{G}$ and $H$ is isomorphic to a subgraph of $G_n$. A second question is to define a strategy for the Builder to win for given $\mathcal{G}, G$ and $H$.

This is an independent study proposal. The following text provides some more information about the area of interest, outlines a specific problem, provides the method and schedule for the study and lists intended deliverables. It ends by providing a suggestion of method to evaluate the work done. The sponsor of this Independent Study is Dr. Stanisław P. Radziszowski. The credits of this Independent Study will count towards Advanced Electives.