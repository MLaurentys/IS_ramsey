\subsection*{Context}

The planar graphs form an important class of graphs and the result of considering the Online Ramsey Games with the class $\mathcal{G} = \{G \;|\; G \text{ is planar}\}$ is a focus of study as well. In the past few years many facts were proved. For example, Grytczuk, Haluszczak and Kierstead proved that every forest in the class of forests (a subclass of planar graphs) is a win for the Builder \cite{1}. In the same text, the authors conjectured that ``The class of graphs unavoidable on planar graphs is exactly the class of outerplanar graphs".

Ten years later in 2014, the conjecture was proved wrong by Sarka Petrickova \cite{2}. She showed that there is an entire class of planar graphs, that are not outerplanar, which is unavoidable. In the following year, Hojin Choi showed in a master thesis \cite{3,4} that the Painter wins the game with target $K_3$ on the class of $K_4$-minor-free graphs (another subclass of the planar graphs). There are still many conjectures open and new results come up almost every other year.

Just  like with the Ramsey Numbers counter-part, progress is made both in the algebraic part of the theory and in the computational part. In 2018, Przemysław Gordinowicz and Paweł Prałat\cite{5} showed a new method for calculating Online Ramsey Numbers that increased performance for a special case by a factor of $2\cdot10^6$. That is not specific to the planar version but serves to show that this is another area under active development.

For this Independent Study project, the topic of interest is taking advantage of the fact the graphs discussed can be embedded in a plane by generating visual representations of Builder and Painter games and strategies. At times, these strategies are used in proofs for theorems, which will be the focus of the visualizations generated.

The idea is not only making them more accessible, but also help finding patterns between different proofs for different theorems. Additionally, visualization is a valid method of identifying problematic strategies and allowing human interaction, which are important for testing purposes. A software that generates such visualization would be a great tool for future contributions to the field.